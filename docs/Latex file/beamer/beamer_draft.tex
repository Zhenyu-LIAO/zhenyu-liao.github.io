\documentclass[10pt]{beamer}
%\documentclass[10pt,notes=show]{beamer}
%\documentclass[10pt,handout,english,notes=show]{beamer}
\usepackage[english]{babel} %possibile for french if needed
%\usepackage{etex}
%\usepackage{pgfpages}
\usepackage{color,hyperref}
\usepackage{graphicx}
%\usepackage{longtable}

%\usepackage{pifont}
\usepackage{amsmath}
\usepackage{stmaryrd}
\usepackage{amsthm}
\usepackage[utf8]{inputenc}
%use for checkmark,etc
\usepackage{pifont}
%usepackage{tikz,pgfplots}
%\usetikzlibrary{spy}
%\usepackage[backend=biber,style=numeric-comp,sorting=none]{biblatex}
%\addbibresource{biblio.bib}


%\setbeamerfont{title}{family=\rm}

\mode<presentation>
{
  \setbeamertemplate{footline}[frame number]{}
  \setbeamertemplate{navigation symbols}{}
  \setbeamertemplate{frametitle continuation}{\frametitle{}}
  \usefonttheme[onlymath]{serif}
  \usetheme{default}
  \usecolortheme{beaver}
  \setbeamercovered{transparent}
}


% title information
\title[Crisis] % (optional, only for long titles)
{The Economics of Financial Crisis}
\subtitle{Evidence from India}
\author[Author, Anders] % (optional, for multiple authors)
{F.~Author\inst{1} \and S.~Anders\inst{2}}
\institute[Universities Here and There] % (optional)
{
  \inst{1}%
  Institute of Computer Science\\
  University Here
  \and
  \inst{2}%
  Institute of Theoretical Philosophy\\
  University There
}
\date[KPT 2004] % (optional)
{Conference on Presentation Techniques, 2004}
\subject{Computer Science}



% main document
\begin{document}

%% PDF settings
\hypersetup{pdfstartview={Fit}}




% title page
\frame{\titlepage}
\addtocounter{framenumber}{-1}

% TOC
\begin{frame}
\frametitle{Table of Contents}
\tableofcontents[currentsection,currentsubsection]
\end{frame}

 \AtBeginSection[]
 {
   \begin{frame}<beamer>{Outline}
     \tableofcontents[currentsection,currentsubsection]
   \end{frame}
 }
 \AtBeginSubsection[]
 {
   \begin{frame}<beamer>{Outline}
     \tableofcontents[currentsection,currentsubsection]
   \end{frame}
 }
 \AtBeginSubsubsection[]
 {
   \begin{frame}<beamer>{Outline}
     \tableofcontents[currentsection,currentsubsection]
   \end{frame}
 }
%\beamerdefaultoverlayspecification{<+->}


\section[Section]{My section}
\subsection[Subsection]{My subsection}
\subsubsection[Subsubsection]{My subsubsection}

\begin{frame}
    \frametitle{This is the first slide}
    %Content goes here
  \end{frame}
  \begin{frame}
    \frametitle{This is the second slide}
    \framesubtitle{A bit more information about this}
    %More content goes here
  \end{frame}

\section{Motivation}

\subsection{The Basic Problem That We Studied}

\begin{frame}{Make Titles Informative. Use Uppercase Letters.}{Subtitles are optional.}
  % - A title should summarize the slide in an understandable fashion
  %   for anyone how does not follow everything on the slide itself.

  \begin{itemize}
  \item
    Use \texttt{itemize} a lot.
  \item
    Use very short sentences or short phrases.
  \end{itemize}
\end{frame}

\begin{frame}{Make Titles Informative.}

  You can create overlays\dots
  \begin{itemize}
  \item using the \texttt{pause} command:
    \begin{itemize}
    \item
      First item.
      \pause
    \item    
      Second item.
    \end{itemize}
  \item
    using overlay specifications:
    \begin{itemize}
    \item<3->
      First item.
    \item<4->
      Second item.
    \end{itemize}
  \item
    using the general \texttt{uncover} command:
    \begin{itemize}
      \uncover<5->{\item
        First item.}
      \uncover<6->{\item
        Second item.}
    \end{itemize}
  \end{itemize}
\end{frame}

%% Some animation
 \begin{frame}
\frametitle{Some background}
We start our discussion with some concepts.
\pause
The first concept we introduce originates with Erd\H os.
\end{frame}

%% Text animation
\begin{frame}
\begin{itemize}
  \item This one is always shown
  \item<1-> The first time (i.e. as soon as the slide loads)
  \item<2-> The second time
  \item<1-> Also the first time
  \only<1-1> {This one is shown at the first time, but it will hide soon (on the next event after the slide loads).}
\end{itemize}
\end{frame}
%% alternative way
\begin{frame}
  \frametitle{`Hidden higher-order concepts?'}
  \begin{itemize}[<+->]
  \item The truths of arithmetic which are independent of PA in some 
  sense themselves `{contain} essentially {\color{blue}{hidden higher-order}},
   or infinitary, concepts'???
  \item `Truths in the language of arithmetic which \ldots
  \item That suggests stronger version of Isaacson's thesis. 
  \end{itemize}
\end{frame}
\note{Here is some note.}

% % Example of \footfullcie
% \begin{frame}
%  \frametitle{Title}
%  A reference~\footfullcite{ref_bib}, with ref_bib an item of the .bib file.
% \end{frame}

% \begin{frame}[fragile]
% \frametitle{Source code}

% \begin{lstlisting}[caption=First C example]
% int main()
% {
%     printf("Hello World!");
%     return 0;
% }
% \end{lstlisting}
% \end{frame}

%%Columns 
\begin{frame}{Example of columns 1}
    \begin{columns}[c] % the "c" option specifies center vertical alignment
    \column{.5\textwidth} % column designated by a command
     Contents of the first column
    \column{.5\textwidth}
     Contents split \\ into two lines
    \end{columns}
\end{frame}

\begin{frame}{Example of columns 2}
     \begin{columns}[T] % contents are top vertically aligned
     \begin{column}[T]{5cm} % each column can also be its own environment
     Contents of first column \\ split into two lines
     \end{column}
     \begin{column}[T]{5cm} % alternative top-align that's better for graphics
          \begin{figure}
          \centering{}\includegraphics[height=3cm]{CONVEX_N-P-SPLITTING.pdf}\caption{Illustration of projection algorithm}
          \end{figure}
     \end{column}
     \end{columns}
\end{frame}


%%Block 
\begin{frame}

   \begin{block}{This is a Block}
      This is important information
   \end{block}

   \begin{alertblock}{This is an Alert block}
   This is an important alert
   \end{alertblock}

   \begin{exampleblock}{This is an Example block}
   This is an example 
   \end{exampleblock}

\end{frame}


\section*{Summary}

\begin{frame}{Summary}

  % Keep the summary *very short*.
  \begin{itemize}
  \item
    The \alert{first main message} of your talk in one or two lines.
  \item
    The \alert{second main message} of your talk in one or two lines.
  \item
    Perhaps a \alert{third message}, but not more than that.
  \end{itemize}
  
  % The following outlook is optional.
  \vskip0pt plus.5fill
  \begin{itemize}
  \item
    Outlook
    \begin{itemize}
    \item
      Something you haven't solved.
    \item
      Something else you haven't solved.
    \end{itemize}
  \end{itemize}
\end{frame}



% All of the following is optional and typically not needed. 
\appendix
\section<presentation>*{\appendixname}
\subsection<presentation>*{For Further Reading}

\begin{frame}[allowframebreaks]
  \frametitle<presentation>{For Further Reading}
    
  \begin{thebibliography}{10}
    
  \beamertemplatebookbibitems
  % Start with overview books.

  \bibitem{Author1990}
    A.~Author.
    \newblock {\em Handbook of Everything}.
    \newblock Some Press, 1990.
 
    
  \beamertemplatearticlebibitems
  % Followed by interesting articles. Keep the list short. 

  \bibitem{Someone2000}
    S.~Someone.
    \newblock On this and that.
    \newblock {\em Journal of This and That}, 2(1):50--100,
    2000.
  \end{thebibliography}
\end{frame}

%% Bibliography
\begin{frame}[allowframebreaks]
  \frametitle<presentation>{Further Reading}    
  \begin{thebibliography}{10}    
  \beamertemplatebookbibitems
  \bibitem{Autor1990}
    A.~Autor.
    \newblock {\em Introduction to Giving Presentations}.
    \newblock Klein-Verlag, 1990.
  \beamertemplatearticlebibitems
  \bibitem{Jemand2000}
    S.~Jemand.
    \newblock On this and that.
    \newblock {\em Journal of This and That}, 2(1):50--100, 2000.
  \end{thebibliography}
\end{frame}

\end{document}