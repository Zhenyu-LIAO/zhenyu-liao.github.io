%
% As physicist we often submit to physical review journals which have
% the "revtex" environment as default. I therefore use revtex here.
% revtex is however very similar to the standard "article" documentclass environment so
% that there should be no compatibility issues.
%
% There are many options you can choose (in [  ] brackets) - see the revtex manual from
% the american physics society
%
\documentclass[aps,pre,preprint,nofootinbib]{revtex4}

% You should use BibTeX and apsrev.bst for references
% Choosing a journal automatically selects the correct APS
% BibTeX style file (bst file), so only uncomment the line
% below if necessary.
%\bibliographystyle{apsrev} % standard revtex style 
%\bibliographystyle{plain} % shows titles
%
% apsrevlong.bst: Nonstandard style hacked by L. Illing to give journal titles
% but otherwise conform to apsrev bibliography style.
% You need to obtain "apsrevlong.bst" from the course webpage or me. 
% It is not contained in any standard LaTeX distribution.
\bibliographystyle{apsrevlong}

% If your document is named foo.tex and if you use BibTeX, then you need to invoke
%   latex foo
%   bibtex foo
%   latex foo
%   latex foo
% to get the correct document out in the end. You may use 'pdflatex foo' instead of
% 'latex foo' to obtain a pdf file instead of a dvi file. (Note that in TeXShop PdfTeX is used by default)

%
% some useful packages
%
\usepackage{amsmath,amssymb,amsfonts,amsthm}
\usepackage{graphicx}
\usepackage{bbm}


  % The switch below allows us to put the .pdf figures needed when 'pdflatex' is
  % used and the .eps figured required when 'latex' is used into different 
  % directories. The path to these directories is specified here and does not
  % have to be specified in the text
  %\ifx\pdfoutput\undefined
  %    \usepackage{graphicx}
  %    \graphicspath{{../epsfigs/}}	
  % \else
  %    \usepackage[pdftex]{graphicx}
  %    \graphicspath{{../pdffigs/}}	
  % \fi


\begin{document}
%%%%%%%%%%%%%%%%%%%%%%%%%%%%%%%%%%%%%%%%%%%%%
% Definitions
%
%
% Define your special symbols here
%
%%%%%%%%%%%%%%%%%%%%%%%%%%%%%%%%%%%%%%%%%%%%%

%%%%%%%%%%%%%%%%%%%%%%%%%%%%%%%%%%%%%%%%%%%%%
% End Definitions
%%%%%%%%%%%%%%%%%%%%%%%%%%%%%%%%%%%%%%%%%%%%%


%Title of paper
\title{title}


% Place the author information here.  Please hand-code the contact
% information and notecalls; do *not* use \footnote commands.  Let the
% author contact information appear immediately below the author names
% as shown.  We would also prefer that you don't change the type-size
% settings shown here.

% repeat the \author .. \affiliation  etc. as needed
% \email, \thanks, \homepage, \altaffiliation all apply to the current
% author. Explanatory text should go in the []'s, actual e-mail
% address or url should go in the {}'s for \email and \homepage.
% Please use the appropriate macro foreach each type of information

% \affiliation command applies to all authors since the last
% \affiliation command. The \affiliation command should follow the
% other information
% \affiliation can be followed by \email, \homepage, \thanks as well.

\author{You}
%\affiliation{Department of Physics, Reed College, Portland, Oregon,  97202, USA}
%\email[]{illing@reed.edu}
%\homepage[]{Your web page}
%\thanks{}
%\altaffiliation{}

\author{Lab Partner: }
\affiliation{Department of Physics, Reed College, Portland, Oregon,  97202, USA}
%\email[]{Your e-mail address}
%\homepage[]{Your web page}
%\thanks{}
%\altaffiliation{}

\date{\today}

\begin{abstract}  
\end{abstract}


%\maketitle must follow title, authors, abstract, \pacs, and \keywords
\maketitle



\section{content}

\subsection{more content}

....


\begin{acknowledgments}
\end{acknowledgments}


\appendix*
\section{}

%
% If you use BibTeX you need to link to your .bib database here
% Say you use foo.bib, then write \bibliography{foo}
%
%\bibliography{}

%
% If you think BibTeX is too much of a pain, you can alternatively hand-code the bibliography. In this case you
% need to make sure that you adhere to the formatting rules of the journal or reed-thesis. 
% If you want to hand code, do not use \bibliogrpahy{} but  \begin{thebibliography}{99} ..... \end{thebibliography}.
% In between insert for each citation \bibitem[label] citation
%
% Example:
%
% 	\begin{thebibliography}{99}
%
%	\bibitem[illing2005]  L. Illing and D. J. Gauthier, ``Hopf bifurcations in time-delay systems with band-limited feedback,'' Physica D {\bf 210}, 180 (2005).
%
%	\end{thebibliography}



\end{document}

















